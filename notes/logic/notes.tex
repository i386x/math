%
% File:    ./notes/logic/notes.tex
% Author:  Jiří Kučera <sanczes AT gmail.com>
% Date:    2022-08-22 13:58:30 +0200
% Project: Math Notes
% Brief:   Notes about mathematical logic
%
% SPDX-License-Identifier: CC-BY-SA-4.0
%

\input macros/notes

\title\MathNotes%
\subject{Mathematical Logic}%
%\author{Ji\v r\'i Ku\v cera}%

\frontcover
\end
\titlepage
\infopage

\chapter *Preface.

The idea behind \emph{Math Notes} originates from the author's need to organize
his notes taken from various textbooks, papers, and lectures and provide them
to the wider audience. The purpose of this document is also to serve as the
author's external memory. Thus, the structure of the content follows these
rules:
\list
  \item There is no clear border between complex and trivial. What is trivial
    now may not be trivial after ten years.
  \item Definitions, theorems, and algorithms are demonstrated on examples
    whenever it becomes useful.
  \item Proofs should be as clear and understandable as possible. This often
    involves an introduction of auxiliary definitions and theorems as well as
    describing the idea of a proof. All of this aims to elimination of
    situations like {\it Where this step comes from?}
\quit
The reader should be able to read this document from the start to its end
without being stuck at one place. The rules above should help to fulfill this
goal. However, since nothing is perfect, the final decision whether the
author's effort has been successful or not is left to the reader.

\contents

\chapter Introduction (intro).

This document contain a notes related to mathematical logic. The content of the
document is organized into three chapters.

\refsec{ch:intro}, gives the outline of this document. The structure of
following chapters are described here.

\refsec{ch:pcalc}, is dedicated to the propositional calculus.

Finally, \refsec{ch:folog}, is dedicated to the first-order logic.

\chapter Propositional Calculus (pcalc).

This chapter summarizes the notes taken about the propositional calculus.

\section Terminology (term).

Let us first define formally some fundamental terminology. We suppose that the
reader is familiar with the basic notions from algebra (see Chapter 1 in
\cite{MN-Algebra} for definitions and used notation).

\definition Truth value (truth-value):
  A \term{truth value} is either 0 or 1. Truth values 0 and 1 are usually
  called \term{false} and \term{true}, respectively.
\quit

\definition Atomic sentence (atom):
  Let $P$ be a countable nonempty set of symbols such that $P \cap \{0, 1\} =
  \emptyset$. A symbol $p \in P$ is said to be an \term{atomic sentence} if
  there exists a mapping $v$ such that $v(p) \in \{0, 1\}$. The mapping $v$ is
  called an \term{evaluation function}.
\quit

\definition Sentence (sentence):
  Let $A$ be a set of atomic sentences and $C_1$ and $C_2$ be two sets of
  symbols such that $A$, $C_1$, $C_2$, and $\{0, 1\}$ are pairwise disjoint.
  Then a \term{sentence} is defined recursively in the following way:
  \list{a}
    \item(1) If $a \in A$, then $a$ is a sentence.
    \item(2) If $a$ is a sentence and $\sigma \in C_1$, then $(\sigma a)$ is a
      sentence.
    \item(3) If $a$ and $b$ are sentences and $\circ \in C_2$, then $(a \circ
      b)$ is a sentence.
    \item Every sentence is made only by a finite number of applications of
      rules \refitem{1}, \refitem{2}, and \refitem{3}.
  \quit
  Symbols from $C_1$ and $C_2$ are said to be unary and binary
  \term{propositional connectives}, respectively. The application of rules
  \refitem{list/1} and \refitem{list/2} is called a \term{truth-functional
  combination}. Let $c$ be a propositional connective that appears in the last
  application of either rule \refitem{list/1} or rule \refitem{list/2}. Then
  $c$ is said to be the \term{principal connective}.
\quit

\definition Truth function (truth-function):
  Let $n$ be a positive integer. A \term{truth function} is a mapping from
  $\{0, 1\}^n$ to $\{0, 1\}$.
\quit

\definition Truth table (truth-table):
  Let $k$ be a positive integer. Let $f_i$ be a truth function of $n_i$
  arguments, $n_i \ge 1$, $1 \le i \le k$. Set $n_{\rm max} = \max\{n_i \where
  1 \le i \le k\}$. Let $\pi_i$ be a mapping from $\{j \where 1 \le j \le n_i
  \}$ to $\{j \where 1 \le j \le n_{\rm max}\}$, $1 \le i \le k$. The
  \term{truth table} for $f_i$, $1 \le i \le k$, is constructed in the
  following way:
  \list{1}
    \item Let $\tau_{ij}$ denote the value at $i$th row and $j$th column, $1
      \le i \le 2^{n_{\rm max}}$, $1 \le j \le (n_{\rm max} + k)$.
    \item $\sum_{j = 1}^{n_{\rm max}} \tau_{ij} 2^{n_{\rm max} - j} = i - 1$,
      $1 \le i \le 2^{n_{\rm max}}$.
    \item $f_l(\tau_{i\pi_l(1)}, \tau_{i\pi_l(2)}, \ldots, \tau_{i\pi_l(n_l)})
      = \tau_{i(l + n_{\rm max})}$, $1 \le l \le k$, $1 \le i \le 2^{n_{\rm
      max}}$.
  \quit
\quit

\definition Propositional connectives (pconns):
  Let $S$ be a set of sentences and $A \subset S$ be a set of atomic sentences.
  Define propositional connectives $\lnot$, $\land$, $\lor$, $\imply$, $\iff$,
  and $\lxor$ and extend evaluation function from $A$ to $S$ in the following
  way:
  \list
    \item For every $x \in S$, $(\lnot x) \in S$ is said to be the
      \term{negation} of $x$. If $v(x) = 0$, then $v(\lnot x) = 1$. Otherwise,
      $v(\lnot x) = 0$.
    \item For every $x, y \in S$, $(x\land y) \in S$ is said to be the
      \term{conjunction} of $x$ and $y$, where $x$ and $y$ are called
      \term{conjuncts}. If $v(x) = 1$ and $v(y) = 1$, then $v(x\land y) = 1$.
      Otherwise, $v(x\land y) = 0$.
    \item For every $x, y \in S$, $(x\lor y) \in S$ is said to be the
      \term{disjunction}, or \term{inclusive or}, of $x$ and $y$, where $x$ and
      $y$ are called \term{disjuncts}. If $v(x) = 0$ and $v(y) = 0$, then $v(x
      \lor y) = 0$. Otherwise, $v(x\lor y) = 1$.
    \item For every $x, y \in S$, $(x\imply y) \in S$ is said to be a
      \term{conditional}, or \term{implication}, where $x$ is called the
      \term{antecedent} and $y$ is called the \term{consequent}. If $v(x) = 1$
      and $v(y) = 0$, then $v(x\imply y) = 0$. Otherwise, $v(x\imply y) = 1$.
    \item For every $x, y \in S$, $(x\iff y) \in S$ is said to be a
      \term{biconditional} or \term{logical equivalence}. If $v(x) = v(y)$,
      then $v(x\iff y) = 1$. Otherwise, $v(x\iff y) = 0$.
    \item For every $x, y \in S$, $(x\lxor y) \in S$ is said to be a
      \term{exclusive or} or \term{logical non-equivalence}. If $v(x) \neq v(y)
      $, then $v(x\iff y) = 1$. Otherwise, $v(x\iff y) = 0$.
  \quit
\quit

Given an evaluation function, a sentence has a constant value. In the next
definition, we introduce notions {\it statement letter} and {\it statement
form} for a symbol that can represent an arbitrary sentence and logical
expression made upon these symbols, respectively.

\definition Statement form (stmt-form):
  Let $S$ be a countable nonempty set of symbols. A symbol $A \in S$ is said to
  be a \term{statement letter} if it can represent an arbitrary sentence. A
  \term{statement form} is defined recursively in the following way:
  \list{1}
    \item(1) For every $A \in S$, $A$ is a statement form.
    \item(2) For every $A, B \in S$, $(\lnot A)$, $(A\land B)$, $(A\lor B)$,
      $(A\imply B)$, $(A\iff B)$, and $(A\lxor B)$ are statement forms.
    \item Every statement form is made only by a finite number of applications
      of rules \refitem{1} and \refitem{2}.
  \quit
\quit

\docend
